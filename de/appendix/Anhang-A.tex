%===========================
% Anhang A: Mathematische Grundlagen
%===========================


\appendix
\renewcommand{\thesection}{A.\arabic{section}}

\renewcommand{\thechapter}{A}
\chapter{Mathematische Grundlagen}
\label{anhangA_mathe}



\noindent
Die Kapitel des Hauptteils haben gezeigt, wie sich der Zahlbegriff im Laufe der Geschichte 
erweitert hat – von den natürlichen Zahlen bis zu den komplexen Zahlen. 
Um den Lesefluss nicht zu überfrachten, wurden die mathematischen Details dabei 
meist nur skizziert. 

In diesem Anhang finden sich einige der klassischen Herleitungen, die für das Verständnis 
der Entwicklung des Zahlbegriffs zentral sind. Sie zeigen, wie Logik und Beweisführung 
neue Zahlarten notwendig machten. 


Damit dient dieser Anhang als Ergänzung: Er bietet die mathematische Tiefe für alle, 
die genauer verstehen möchten, wie die neuen Zahlbereiche Schritt für Schritt 
logisch begründet wurden.

\newpage
\refstepcounter{section}\label{anhangA_wurzel2}
\section*{A.1 Widerspruchsbeweis für die Irrationalität von $\sqrt{2}$}
\addcontentsline{toc}{section}{A.1 Widerspruchsbeweis für die Irrationalität von $\sqrt{2}$}
\label{anhangA_wurzel2}

\noindent
% --- A.1 Beweis der Irrationalität von sqrt(2) ---



\subsubsection*{Vorbemerkung (Lemma)}
\phantomsection
\textbf{Lemma.} Ist $p^2$ eine \emph{gerade} Zahl, so ist auch $p$ gerade.

\emph{Begründung.} Wäre $p$ ungerade, dann $p=2k+1$ und damit
$p^2=(2k+1)^2=4k^2+4k+1=2(2k^2+2k)+1$ ungerade — Widerspruch. \hfill$\square$

\vspace{0.5em}

\subsubsection*{Beweis (Widerspruchsbeweis)}
\phantomsection
Angenommen, $\sqrt{2}$ sei \emph{rational}. Dann existieren ganze Zahlen $p,q$ ohne gemeinsamen Teiler
($\gcd(p,q)=1$) mit
\[
\sqrt{2}=\frac{p}{q}.
\]
Quadrieren liefert
\[
2=\frac{p^2}{q^2}\quad\Longrightarrow\quad p^2=2q^2.
\]
Damit ist $p^2$ gerade, also nach dem Lemma auch $p$ gerade. Schreibe $p=2r$.
Dann folgt
\[
p^2=(2r)^2=4r^2=2q^2 \;\Longrightarrow\; q^2=2r^2,
\]
also ist auch $q^2$ gerade und damit $q$ gerade.

Damit sind \emph{beide} Zahlen $p$ und $q$ gerade — sie besitzen also den gemeinsamen Teiler $2$.
Das widerspricht der getroffenen Annahme, dass $p$ und $q$ teilerfremd sind. 
Die Ausgangsannahme war folglich falsch, also ist $\sqrt{2}$ \emph{irrational}. \hfill$\square$

\DidaktikBox{Was hier passiert (Idee des Widerspruchs)}{box:idee_wurzel2_lang}{%
	Wir nehmen an, $\sqrt{2}$ sei \emph{doch} eine Bruchzahl in vollständig gekürzter Form $\frac{p}{q}$.
	Aus der Gleichung $p^2=2q^2$ folgt nacheinander, dass $p$ gerade ist und \emph{deshalb} auch $q$ gerade sein muss.
	Damit wäre $\frac{p}{q}$ aber \emph{nicht} vollständig gekürzt. Genau dieser \glqq Teilerfremdheits-Widerspruch\grqq{}
	zeigt: Die Annahme war unmöglich — also ist $\sqrt{2}$ irrational.
}

\HinweisBox{Alternative Sichtweisen}{box:alternativen_wurzel2}{%
	\textbf{(1) Primfaktorzerlegung:} In der eindeutigen Primfaktorzerlegung besitzt $2$ einen \emph{ungeraden}
	Exponenten, während ein Quadrat stets nur \emph{gerade} Exponenten hat — das passt nicht zusammen.\\[0.3em]
	\textbf{(2) Kettenbruch:} Die Darstellung $\sqrt{2}=[1;\overline{2}]$ ist \emph{periodisch unendlich};
	solche Kettenbrüche entsprechen irrationalen Zahlen.
}

\newpage
\refstepcounter{section}\label{anhangA_i}
\section*{A.2 Von negativen Wurzeln zu $i^2=-1$}
\addcontentsline{toc}{section}{A.2 Von negativen Wurzeln zu $i^2=-1$}
\label{anhangA_i}

\noindent
Im 16.~Jahrhundert stießen italienische Mathematiker bei der Lösung kubischer Gleichungen 
auf Ausdrücke wie $\sqrt{-1}$.  
Zunächst galten diese als „unmöglich“, doch \textbf{Rafael (Raffaele) Bombelli}\index{Bombelli, Rafael} 
zeigte, dass man sie konsistent behandeln kann, wenn man $i^2=-1$ setzt.  

\MatheBox{Definition der imaginären Einheit}{box:anhang_i}{%
	Die imaginäre Einheit $i$ wird definiert durch
	\[
	i^2 = -1.
	\]
	Jede komplexe Zahl lässt sich dann als 
	\[
	z = a + bi \quad \text{mit } a,b \in \mathbb{R}
	\]
	schreiben. 
}

\noindent
Dieser Schritt führte zu einer völlig neuen Zahlenwelt. 
Zunächst misstrauisch betrachtet, wurden komplexe Zahlen später durch 
Euler, Gauss und Argand in der Mathematik fest verankert. 
Heute bilden sie eine unverzichtbare Grundlage vieler Gebiete der Mathematik und Physik.
