
\cleardoublepage
\appendix
\renewcommand{\thechapter}{A}

\renewcommand{\thesection}{\Alph{chapter}.\arabic{section}}

\chapter{Mathematische Hintergründe und Herleitungen}
\label{anhangA}

In diesem Anhang werden die im Haupttext angesprochenen physikalischen Konzepte
formal und mathematisch vertieft. Ziel ist es, die didaktische Lesbarkeit der
Kapitel nicht zu beeinträchtigen und zugleich interessierten Lesern die
vollständigen Herleitungen zugänglich zu machen. 

Die Abschnitte sind thematisch nach den zentralen Eigenschaften des Photons
gegliedert, darunter Energie-Impuls-Relation, Massehypothese, Helizität und
Polarisation.\index{Photon!Eigenschaften} Auf diese Weise bildet der Anhang eine Brücke zwischen den
intuitiven Erklärungen im Haupttext und der mathematischen Strenge der
Quantenfeldtheorie.

%\section{Kapitel I}
\phantomsection
\section{Energie- und Impulsrelation des Photons}\index{Energie-Impuls-Relation!Photon}
\label{anhangA:energie_impuls}

In diesem Abschnitt wird formal hergeleitet, warum ein Photon die Energie
