\chapter{Einleitung: Warum Mathematik Struktur ist}
\label{chap:I_einfuehrung}

\setlength{\parindent}{0pt}


\subsection{Ausgangsfrage}
\index{Mathematik}
\index{Entdeckung vs. Erfindung}
\index{Naturwissenschaften}
\index{Galilei, Galileo}

Warum brauchen wir überhaupt \textbf{Zahlen}, \textbf{Formen} und \textbf{Funktionen}, um die Welt zu beschreiben?
Ist \textbf{Mathematik} nur ein Werkzeug, das Menschen erfunden haben, oder steckt dahinter eine tiefere Struktur, die wir entdecken?
Diese Frage begleitet die gesamte Geschichte der Naturwissenschaften.

Schon die Griechen erkannten, dass Harmonie -- in Musik, Geometrie und sogar im Kosmos -- auf Zahlenverhältnissen beruht.
Später zeigte \textbf{Galileo Galilei}, dass die Natur „in der Sprache der Mathematik“ geschrieben sei.
Und bis heute bleibt es erstaunlich, dass abstrakte Symbole, die zunächst nur im Denken existieren,
plötzlich die Bahnen von Planeten, die Ausbreitung des Lichts oder die Gesetze der Quantenmechanik beschreiben.

Die Ausgangsfrage ist also nicht nur theoretisch, sondern berührt unser Bild von der Wirklichkeit:
Entsteht Mathematik in unserem Denken, oder spiegelt sie eine Ordnung wider, die in der Natur selbst vorhanden ist?
\subsection{Mathematik als universelle Sprache}
\label{sec:1.2_universelle_sprache}
\index{universelle Sprache}
\index{Kulturen}
\index{Null}
\index{Algebra}
\index{irrationale Zahlen}

Auffällig ist, dass mathematische Strukturen unabhängig von Kultur und Zeit immer wieder auftauchen.
Die Griechen stießen auf \emph{irrationale Zahlen}, die Inder führten die \emph{Null} ein,
arabische Mathematiker entwickelten die \emph{Algebra}.
Trotz unterschiedlicher Sprachen und Traditionen zeigt sich eine gemeinsame innere Logik.

Besonders bemerkenswert ist, dass verschiedene Kulturen, oft völlig unabhängig voneinander,
zu denselben mathematischen Ergebnissen gelangten -- sei es bei der Berechnung von Quadratwurzeln,
bei der Entwicklung von Zahlensystemen oder bei geometrischen Sätzen.
Das spricht gegen eine bloß willkürliche Erfindung:
Mathematik wirkt wie eine Struktur, die immer wieder \emph{entdeckt} wird.

Mathematik erscheint damit als universelle Sprache,
die allen Kulturen zugänglich ist -- und die heute gleichermaßen in Informatik,
Quantenphysik und künstlicher Intelligenz sichtbar wird.

\begin{NoteBox}[Unabhängige Entdeckungen]
	\small
	Viele Kulturen haben zentrale mathematische Ideen unabhängig voneinander gefunden.
	Das spricht dafür, dass Mathematik nicht nur ein Zeichensystem ist,
	sondern eine Struktur, die immer wieder freigelegt wird.
	\label{box:entdeckung}
\end{NoteBox}
\subsection{Experimente als Schlüssel}
\index{Experiment}
\index{Galilei, Galileo}

\index{Thermometer}

Erst durch Experimente zeigte sich, dass diese abstrakten Strukturen in der Natur selbst wirksam sind.
\textbf{Galileo Galilei} ließ Kugeln eine schiefe Ebene hinunterrollen und fand:
Die Bewegung folgt einer Parabel.
Ein Thermometer macht die Linearität der Ausdehnung von Quecksilber sichtbar -- eine gerade Linie im Diagramm.
\textbf{Georg Simon Ohm} maß Strom, Spannung und Widerstand und formulierte die Beziehung
\[
U=R\cdot I.
\]
Nicht das Denken allein, sondern das Experiment offenbart solche Gesetzmäßigkeiten.
\begin{NoteBox}[Experiment als Übersetzer]
	\small
	Das Experiment ist die Brücke zwischen abstrakter Struktur und Wirklichkeit:
	Es macht messbar, welche mathematische Beziehung in einem Phänomen tatsächlich wirksam ist.
\end{NoteBox}
\subsection{Entdeckung statt Erfindung}
\index{Entdeckung vs. Erfindung}
\index{Pi@$\pi$}
\index{Eulersche Zahl $e$}
\index{Platon}
\index{Platonismus}

Viele mathematische Konstanten lassen sich nicht willkürlich festlegen.
Die Kreiszahl $\pi$ ergibt sich zwangsläufig aus der Geometrie des Kreises,
die Eulersche Zahl $e$ taucht unvermeidlich in Wachstumsprozessen und Grenzwerten auf.
Solche Zahlen und Strukturen sind nicht frei ausgedacht:
Wir finden sie als notwendige Konsequenzen definierter Begriffe und Beziehungen wieder.

Schon \textbf{Platon} sah in der Mathematik eine Ideenwelt, die unabhängig vom Menschen existiert.
Bis heute wird diskutiert, ob Mathematik erfunden oder entdeckt ist --
eine Frage, die Philosophie und Naturwissenschaft verbindet.

\subsection{Fazit}

Die Basis jeder Mathematik sind die Zahlen.
Von der einfachen Zählung über irrationale und negative Zahlen bis hin zu komplexen Zahlen zeigt sich:
Jede Erweiterung öffnet einen tieferen Blick auf Struktur.

Dieses Buch verfolgt eine klare Leitidee:
Mathematik ist nicht bloß erfunden, sondern wird als universelle Struktur immer wieder entdeckt.
Diese Struktur ist Kultur und Zeit nicht unterworfen -- sie ist prinzipiell allen zugänglich,
die sich auf Beweise, Begriffe und Konsequenzen einlassen.

\begin{NoteBox}[Zahlen als universale Entdeckung]
	\small
	Die Geschichte zeigt: Ob Babylonier, Inder, Griechen oder Araber --
	verschiedene Kulturen stießen auf dieselben mathematischen Grundideen.
	Den Anfang bilden stets die Zahlen.
	Darum setzt Kapitel II genau hier an: bei den fundamentalen Bausteinen der Mathematik.
	\label{box:universale}
\end{NoteBox}