\chapter{Einleitung: Warum Mathematik Struktur ist}
\label{chap:I_einfuehrung}

\setlength{\parindent}{0pt}


\subsection{Ausgangsfrage}

Warum brauchen wir überhaupt \textbf{Zahlen}\index{Zahlen}, \textbf{Formen}\index{Formen} und \textbf{Funktionen}\index{Funktionen}, um die Welt zu beschreiben? 
Ist \textbf{Mathematik}\index{Mathematik} nur ein Werkzeug, das Menschen erfunden haben oder steckt dahinter eine tiefere Struktur, die wir entdecken? 
Diese Frage begleitet die gesamte Geschichte der \textbf{Naturwissenschaften}\index{Naturwissenschaften}.
Schon die Griechen erkannten, dass Harmonie – in Musik, Geometrie \index{Geometrie} und sogar im Kosmos \index{Kosmos} – auf Zahlenverhältnissen beruht.
\index{Harmonie}
Später zeigte \textbf{Galileo Galilei}\index{Galilei, Galileo}, dass die Natur „in der Sprache der Mathematik“ geschrieben sei.  
Und bis heute bleibt es erstaunlich, dass abstrakte Symbole, die zunächst nur im Kopf existieren, 
plötzlich die Bahnen von \textbf{Planeten}\index{Planeten}, die Ausbreitung des \textbf{Lichts}\index{Licht} oder die Gesetze der \textbf{Quantenmechanik}\index{Quantenmechanik} beschreiben. 

Die Ausgangsfrage ist also nicht nur theoretisch, sondern berührt unser Bild von der Wirklichkeit: 
Entsteht Mathematik in unserem Denken  oder spiegelt sie eine Ordnung wider, die in der Natur selbst vorhanden ist?


\subsection{Mathematik als universelle Sprache}
\label{sec:1.2_universelle_sprache}

Auffällig ist, dass mathematische Strukturen\index{mathematische Strukturen} 
unabhängig von Kultur\index{Kulturen} und Zeit immer wieder auftauchen. 
Die Griechen\index{Griechen} stießen auf \emph{irrationale Zahlen}\index{irrationale Zahlen}, 
die Inder\index{Indien} erfanden die \emph{Null}\index{Null}, 
arabische Mathematiker\index{arabische Mathematik} entwickelten die \emph{Algebra}\index{Algebra}. 
Trotz unterschiedlicher Sprachen und Traditionen ergibt sich eine gemeinsame innere Logik. 

Besonders bemerkenswert ist, dass verschiedene Kulturen, oft völlig unabhängig voneinander, 
auf dieselben mathematischen Ergebnisse gestoßen sind – sei es bei der 
Berechnung von \emph{Quadratwurzeln}\index{Quadratwurzeln}, 
bei der Entwicklung von \emph{Zahlensystemen}\index{Zahlensysteme} 
oder bei geometrischen Sätzen\index{Geometrie}. 
Dies zeigt: Mathematik ist keine willkürliche Erfindung, sondern eine Struktur, 
die überall entdeckt wird.

Mathematik wirkt wie eine universelle Sprache\index{universelle Sprache}, 
die allen Kulturen zugänglich ist – und die heute gleichermaßen in 
\emph{Informatik}\index{Informatik}, \emph{Quantenphysik}\index{Quantenphysik} 
und \emph{künstlicher Intelligenz}\index{künstliche Intelligenz} sichtbar wird.
\vspace{1em}

\begin{NoteBox}[Unabhängige Entdeckungen]
	Viele Kulturen haben zentrale mathematische Ideen unabhängig voneinander gefunden. 
	Das zeigt: Mathematik ist keine Erfindung, sondern eine Struktur, 
	die überall entdeckt wird.
	\label{box:entdeckung}
\end{NoteBox}

\subsection{Experimente als Schlüssel}

Erst durch \textbf{Experimente}\index{Experimente} zeigte sich, dass diese abstrakten Strukturen in der Natur selbst wirksam sind.
\textbf{Galilei, Galileo}\index{Galilei, Galileo} ließ Kugeln eine schiefe Ebene hinunterrollen und fand: Die Bewegung folgt einer \textbf{Parabel}\index{Parabel}. 
Das \textbf{Thermometer}\index{Thermometer} verdeutlichte die Linearität der Ausdehnung von \textbf{Quecksilber}\index{Quecksilber} – eine gerade Linie im Diagramm. 
\textbf{Ohm, Georg Simon}\index{Ohm, Georg Simon} maß Strom, Spannung und Widerstand und entdeckte die einfache Beziehung \( U = R \cdot I \)\index{Ohmsches Gesetz}. 
Nicht das Denken allein, sondern das Experiment offenbarte diese Gesetzmäßigkeiten.

\subsection{Entdeckung statt Erfindung}

Viele mathematische Konstanten lassen sich nicht willkürlich festlegen. 
Die \textbf{Kreiszahl $\pi$}\index{Kreiszahl $\pi$} ergibt sich zwangsläufig aus der Geometrie des Kreises. 
Die \textbf{Eulersche Zahl $e$}\index{Eulersche Zahl $e$} taucht unvermeidlich in Wachstumsprozessen auf. 
Solche Zahlen und Strukturen sind nicht von Menschen ausgedacht, sondern in der Natur angelegt – wir entdecken sie nur. 
Schon \textbf{Platon}\index{Platon} sah in der Mathematik eine Ideenwelt, die unabhängig vom Menschen existiert. 
Bis heute wird diskutiert, ob Mathematik erfunden oder entdeckt ist – eine Frage, die Philosophie und Naturwissenschaft verbindet.  
\subsection{Fazit}

Die Basis jeder \emph{Mathematik}\index{Mathematik} sind die 
\emph{Zahlen}\index{Zahlen}. 
Von der einfachen \emph{Zählung}\index{Zählen} über 
\emph{irrationale Zahlen}\index{irrationale Zahlen} 
und \emph{negative Zahlen}\index{negative Zahlen} 
bis hin zu \emph{komplexen Zahlen}\index{komplexe Zahlen} zeigt sich: 
Jede Erweiterung eröffnet einen tieferen Blick in die 
\emph{Struktur der Welt}\index{Struktur der Welt}.  

In diesem Buch soll die Hypothese\index{Hypothese} aufgezeigt 
und anhand vieler Beispiele belegt werden, 
dass Mathematik nicht erfunden, sondern \emph{entdeckt}\index{Entdeckung} wurde – 
als universelle Struktur\index{universelle Struktur}, 
die allen \emph{Kulturen}\index{Kulturen} und allen Zeiten\index{Zeit} zugänglich ist.  



\vspace{1em}
\begin{NoteBox}[Zahlen als universale Entdeckung]
	Die Geschichte zeigt: Ob \emph{Babylonier}, \emph{Inder}, \emph{Griechen} oder \emph{Araber} – 
	alle Kulturen stießen auf dieselben mathematischen Strukturen. 
	Den Anfang bilden stets die Zahlen. 
	Darum setzt Kapitel II genau hier an: bei den fundamentalen Bausteinen der Mathematik.
	\label{box:universale}
\end{NoteBox}


