
\chapter{Algebra und Gleichungen}
\label{chap:III_algebra}


\subsection{Die arabische Mathematik und die Geburt der Algebra}
\label{sec:3.1_arabische_mathematik}

Nach den Griechen\index{Griechen}, Indern\index{Indien} und Babyloniern\index{Babylonien} 
war es die arabisch-islamische Welt\index{Arabische Mathematik}, 
die die Mathematik\index{Mathematik} auf eine neue, systematische Ebene hob. 
Während in Europa\index{Europa} das antike Wissen weitgehend verloren ging, 
bewahrten und erweiterten Gelehrte wie 
\emph{al-Chwarizmi}\index{al-Chwarizmi, Muhammad ibn Musa}, 
\emph{Omar Chayyām}\index{Chayyām, Omar} und 
\emph{al-Karaji}\index{al-Karaji, Abu Bakr} die Erkenntnisse ihrer Vorgänger. 

Im 9.~Jahrhundert verfasste \emph{al-Chwarizmi} in Bagdad\index{Bagdad} das Werk 
\emph{„Kitāb al-jabr wa’l-muqābala“}\index{Kitāb al-jabr wa’l-muqābala}, 
das als Geburtsstunde der 
\emph{Algebra}\index{Algebra} gilt. 
Der Begriff „al-jabr“ bedeutet sinngemäß „das Wiederherstellen“ oder „das Ergänzen“, 
„al-muqabala“ bezeichnet „das Gegenüberstellen“. 
Gemeint war damit eine Methode, Unbekanntes\index{Unbekannte} zu isolieren und 
Gleichungen\index{Gleichungen} zu vereinfachen – 
eine Revolution im Denken über Zahlen\index{Zahlen} und Formen.
\newpage
\noindent
\begin{HistoryBox}[Von der Praxis zur Theorie]
	Im Mittelpunkt der arabischen Mathematik stand anfangs die Lösung praktischer Probleme: 
	Verteilung von Erbschaften, Berechnung von Grundstücksflächen, 
	Bestimmung von Handelsanteilen oder astronomische Messungen\index{Astronomie}. 
	Doch aus diesen konkreten Aufgaben entwickelte sich ein allgemeines Verfahren, 
	Unbekannte systematisch zu bestimmen – die Geburt der symbolischen Methode.
	\label{box:praxis}
\end{HistoryBox}

Ein klassisches Beispiel aus der frühen arabischen Mathematik ist die Berechnung 
von Erbanteilen\index{Erbanteil}. 
Nach islamischem Recht\index{Islamisches Recht} erhält eine Frau den achten Teil 
des Nachlasses, der Rest wird unter den Söhnen gleich verteilt. 
Beträgt das Erbe insgesamt 600~Dinar, so ergibt sich:

\[
600 = \tfrac{1}{8}\cdot 600 + 2x,
\]

Die Frau erhält daher
\[
\frac{1}{8}\cdot 600 = 75,
\]
und für die zwei Söhne bleiben
\[
600 - 75 = 525.
\]

Damit erhält jeder Sohn
\[
x = \frac{525}{2} = 262{,}5 \text{ Dinar}.
\]

\begin{HistoryBox}[Vom praktischen Problem zur abstrakten Gleichung]
	Die arabischen Mathematiker formulierten alltägliche Aufgaben wie die 
	Berechnung von Erbanteilen in symbolischer Form. 
	
	Ein Beispiel: Aus einem Nachlass von 600~Dinar erhält die Frau ein Achtel. 
	Der Rest wird zu gleichen Teilen an die zwei Söhne verteilt. 
	Das praktische Problem lässt sich direkt in eine Gleichung übersetzen:
	\[
	600 = \frac{1}{8}\cdot 600 + 2x.
	\]
	
	Durch Umformen ergibt sich der Anteil eines Sohnes. 
	Aus solchen realen Situationen entwickelte sich die allgemeine Idee, 
	Unbekannte systematisch zu bestimmen – der Beginn der Algebra als abstraktes Denken.
\end{HistoryBox}

\begin{NoteBox}[Ein gemeinsames Erbe]
	Die arabischen Mathematiker erfanden die Algebra nicht aus dem Nichts. 
	Sie knüpften an das Wissen der Babylonier, Griechen und Inder an. 
	Schon die Babylonier lösten Gleichungen der Form \(x^2 + bx = c\) 
	mit tabellarischen Methoden, und die Griechen beschrieben geometrische Zusammenhänge, 
	die im Kern algebraisch waren. 
	In Indien entstanden symbolische Rechenverfahren für Unbekannte 
	(\emph{yāvattavat}). 
	Die arabischen Gelehrten verbanden all diese Traditionen, 
	gaben ihnen eine einheitliche Sprache und machten daraus ein geschlossenes System – 
	die \emph{al-jabr}, die wir heute Algebra nennen.
	\label{box:erbe}
\end{NoteBox}


\subsection{Al-Chwarizmi: Quadratische Ergänzung als Flächenargument}

Ein bekanntes Beispiel aus Al-Chwarizmis Werk lautet:
\begin{quote}
	„Ein Quadrat und zehn seiner Wurzeln ergeben neununddreißig.“
\end{quote}
Heute schreiben wir dies als Gleichung
\[
x^2 + 10x = 39.
\]

Al-Chwarizmi dachte geometrisch: \(x^2\) ist die Fläche eines Quadrats mit Seitenlänge \(x\).
Der Term \(10x\) ist ebenfalls eine Fläche und kann als Rechteck dargestellt werden.

\vspace{0.6em}

\begin{figure}[htbp]
	\centering
	
	\begin{subfigure}[t]{0.32\linewidth}
		\centering
		\includegraphics[width=\linewidth]{bilder/alch02.png}
		\caption{Flächen: \(x^2\) und \(10x\).}
		\label{fig:alch02}
	\end{subfigure}\hfill
	\begin{subfigure}[t]{0.32\linewidth}
		\centering
		\includegraphics[width=\linewidth]{bilder/alch01.png}
	\caption{Zerlegung: \(10x=2\cdot 5x\).}
		\label{fig:alch01}
	\end{subfigure}\hfill
	\begin{subfigure}[t]{0.32\linewidth}
		\centering
		\includegraphics[width=\linewidth]{bilder/alch03.png}
		\caption{\(+25\Rightarrow (x+5)^2\).}
		\label{fig:alch03}
	\end{subfigure}
	
	\caption{Quadratische Ergänzung nach Al-Chwarizmi als Flächenargument.}
	\label{fig:alchwarizmi}
\end{figure}
	

Den Term \(10x\) zerlegt man als
\[
10x = 5x + 5x,
\]
also als zwei Rechtecke der Größe \(x\times 5\), die man an zwei Seiten des Quadrats anlegt.
Dadurch entsteht fast ein größeres Quadrat; es fehlt nur noch das Eckquadrat \(5\times 5\) mit Fläche \(25\).

Addiert man diese fehlende Fläche auf beiden Seiten, erhält man
\[
x^2 + 10x + 25 = 39 + 25,
\]
also
\[
(x+5)^2 = 64.
\]
Geometrisch bedeutet das: Das ergänzte Quadrat hat die Fläche \(39+25=64\) und damit die Seitenlänge \(8\).
\[
x+5 = \sqrt{64} = 8.
\]
Folglich ist
\[
x = 8 - 5 = 3.
\]



\begin{DidacticBox}[Der Ursprung des Wortes „Algorithmus“]
	Der Name \emph{al-Chwarizmi} wurde im Lateinischen zu „Algoritmi“ verfremdet. 
	Daraus entstand das Wort \emph{Algorithmus}\index{Algorithmus}. 
	Damit geht auch ein zweiter Grundpfeiler der modernen Mathematik – 
	die systematische, regelhafte Berechnung\index{Berechnung} – 
	auf die arabische Wissenschaft zurück.
	\label{box:ursprung}
\end{DidacticBox}

Die Algebra war also weniger eine Entdeckung im modernen Sinn, 
sondern eine geistige Neuordnung des vorhandenen Wissens. 
Sie verband die indischen Zahlensysteme\index{Zahlensystem} 
mit griechischer Geometrie und formte daraus ein allgemeines Rechenverfahren\index{Rechenverfahren}. 
Damit begann die Mathematik, ihre eigene Sprache zu entwickeln – 
eine Sprache der Symbole, Regeln und Abstraktionen\index{Abstraktion}.

% 3.2

\subsection{Gleichungen als Werkzeuge zur Problemlösung}
\label{sec:3.2_gleichungen}
\phantomsection

\subsubsection*{Einleitung}
\phantomsection
Seit dem Altertum\index{Altertum} tauchten in Handel\index{Handel}, 
Geometrie\index{Geometrie} und Astronomie\index{Astronomie} immer wieder
Aufgaben auf, bei denen man eine unbekannte Größe\index{Unbekannte} bestimmen musste.
Anfangs geschah dies mit Worten, Tabellen oder geometrischen Konstruktionen.
Erst allmählich kristallisierte sich eine Methode heraus, die den Umgang mit
dem Unbekannten radikal vereinfachte: die \emph{Gleichung}\index{Gleichung}.

Eine Gleichung macht ein Problem berechenbar.  
Sie übersetzt eine Situation in eine präzise Form, die sich Schritt für Schritt
bearbeiten lässt. Damit wird ein praktisches Problem zu einem mathematischen.

\subsubsection*{Vom Satz zur Gleichung}
\phantomsection
Frühe Gleichungen entstanden meist aus Textaufgaben:
„Ein Händler verkauft die Hälfte seiner Ware plus drei Körbe …“.
Solche Beschreibungen wurden nach und nach durch symbolische
Schreibweisen ersetzt. Besonders die arabische Mathematik\index{Arabische Mathematik} – vertreten durch
\textbf{Muhammad ibn Musa al-Chwarizmi}\index{al-Chwarizmi, Muhammad ibn Musa} –
entwickelte systematische Verfahren, um Gleichungen\index{Gleichung} zu lösen und zu klassifizieren.

Der entscheidende Fortschritt war die Erkenntnis, dass eine Gleichung nicht nur
ein Rechentrick ist, sondern ein Bild der Realität:  
Sie beschreibt Beziehungen\index{Beziehung}. Wer diese Beziehungen beherrscht, kann
unbekannte Größen bestimmen.

\subsubsection*{Gleichungen als universelles Werkzeug}
\phantomsection
Mit der \emph{Algebra}\index{Algebra} entstand ein neuer Zugang zu Problemen:

\begin{itemize}
	\item Geometrische Zusammenhänge konnten algebraisch formuliert werden.
	\item Handels- und Zinsrechnungen\index{Zinsrechnung} ließen sich als 
	lineare Gleichungen\index{Lineare Gleichung} ausdrücken.
	\item Flächen und Volumina führten zu quadratischen\index{Quadratische Gleichung} 
	und kubischen Gleichungen\index{Kubische Gleichung}.
	\item Bewegungsprobleme wurden zu Gleichungen in der Zeit\index{Bewegungsgleichung}.
\end{itemize}

Die Gleichung wurde damit zu einem universellen Werkzeug.
Statt jedes Problem einzeln zu lösen, genügte ein standardisiertes Vorgehen:
Gleichung aufstellen, umformen, Lösung bestimmen.

\subsubsection*{Gleichungen und der erweiterte Zahlbegriff}
\phantomsection
Im 16.~Jahrhundert\index{16. Jahrhundert} zeigte sich, dass Gleichungen nicht nur Lösungen
\emph{finden}, sondern auch Lösungen \emph{fordern}, die außerhalb der bekannten
Zahlenwelt\index{Zahlenwelt} liegen. Damit entstand eine zentrale Einsicht:

\medskip
\textit{Die Algebra bestimmt, welche Zahlen existieren müssen, damit ihre
	Gleichungen lösbar sind.}
\medskip

Quadratische Gleichungen\index{Quadratische Gleichung} erzwangen die 
negativen Zahlen\index{Negative Zahlen}.
Geometrische Konstruktionen führten zur Null\index{Null}.
Kubische Gleichungen\index{Kubische Gleichung} brachten erstmals Ausdrücke wie $\sqrt{-1}$ hervor —
der erste Hinweis auf die späteren komplexen Zahlen\index{Komplexe Zahlen}.

Damit öffnete sich der Weg zu einem völlig neuen Zahlenbereich\index{Zahlenbereich}, 
der sich erst langsam durchsetzte.

\subsubsection*{Ein reales Beispiel}
\phantomsection
Ein einfaches Alltagsproblem zeigt die Denkweise der Algebra besonders klar.
Ein Wasserbehälter verliert durch ein kleines Leck pro Minute
3 Liter Wasser. Zu Beginn enthält er 20 Liter Wasser.  
Wann ist der Behälter leer?

Die Situation lässt sich unmittelbar in eine Gleichung übersetzen:
\[
20 - 3t = 0.
\]

Durch Umformen erhält man:
\[
t = \frac{20}{3} \approx \SI{6.7}{\minute}.
\]

Dieses kleine Beispiel zeigt, wie eine Gleichung eine reale Situation
\emph{formalisiert}\index{Formalisierung}. Die Methode ist universell – unabhängig vom
konkreten Kontext.

\begin{DidacticBox}[Warum Gleichungen die Realität abbilden]
	In vielen Alltagssituationen entsteht eine Gleichung ganz natürlich.
	Die Formulierung fasst die wesentlichen Größen zusammen und
	macht den Zusammenhang berechenbar.
	
	Beim Beispiel des auslaufenden Wasserbehälters:
	\[
	20 - 3t = 0
	\]
	
	erzwingt die Struktur des Problems die Gleichung.
	Die algebraischen Schritte zur Lösung funktionieren für jedes ähnliche
	Problem. Gleichungen sind deshalb ein universelles Werkzeug zur
	Beschreibung von Realität.
\end{DidacticBox}

\subsubsection*{Übergang}
\phantomsection
An diesem Punkt beginnt die eigentliche Geschichte der komplexen Zahlen\index{Komplexe Zahlen}.
Was zunächst wie eine Rechenkuriosität aussah, entwickelte sich zu einer
fundamentalen Erweiterung des Zahlbegriffs\index{Zahlbegriff}.
Die Entstehung dieser neuen Zahlenart wird im nächsten Abschnitt behandelt.



%3.3
\subsection{Die Entstehung der komplexen Zahlen}
\label{sec:3.3_entstehung}
\phantomsection

\subsubsection*{Einleitung}
\phantomsection
Im 16.~Jahrhundert\index{16. Jahrhundert} entwickelten italienische Mathematiker (u.\,a. Tartaglia und Cardano)
Methoden zur Lösung von \emph{Gleichungen dritten Grades}\index{Gleichung dritten Grades} (kubischer Gleichung).
Dabei tauchte ein Ausdruck auf, 
den niemand verstand und den viele für eine Rechenkuriosität hielten: 
die Wurzel einer negativen Zahl\index{Negative Zahlen}. 
Obwohl solche Ausdrücke im Rechenweg auftraten, verschwanden sie im Endergebnis oft wieder. 
Die Mathematiker befanden sich in einer paradoxen Situation: Etwas, das „nicht existieren konnte“, 
war im Lösungsweg unverzichtbar.

Dieser Abschnitt zeigt, wie aus dieser Irritation schrittweise eine der
mächtigsten Erweiterungen der Mathematik\index{Mathematik} entstand: 
die komplexen Zahlen\index{Komplexe Zahlen}.

\subsubsection*{Cardano und das Problem der negativen Wurzeln}
\phantomsection
Gerolamo Cardano\index{Cardano, Gerolamo} veröffentlichte 1545 in seinem Werk 
\emph{Ars Magna}\index{Ars Magna} die erste systematische Darstellung der Lösungen 
kubischer Gleichungen. Bei manchen kubischen Gleichungen (dem \emph{casus irreducibilis}\index{Casus irreducibilis})
erscheinen beim Rechnen Zwischenwerte, die zunächst keinen Sinn zu ergeben scheinen:
\[
\sqrt{-121}\,.
\]
Das ist eine \emph{Wurzel aus einer negativen Zahl} — also der Punkt, an dem die komplexen Zahlen
ins Spiel kommen.
Cardano erkannte klar, dass dieser Ausdruck keinen Sinn ergibt.
Er bezeichnete ihn als \emph{sophistische Wurzel}. Dennoch notierte er
ihn, weil der Rechenweg am Ende das richtige Ergebnis lieferte.

Damit war der Grundstein gelegt: Eine neue Art von Zahl tauchte auf —
noch ohne Bedeutung, aber mathematisch unausweichlich.

\subsubsection*{Bombellis Durchbruch}
\phantomsection
Der italienische Mathematiker Rafael (Raffaele) Bombelli\index{Bombelli, Rafael} erkannte um
1572, dass die merkwürdigen Ausdrücke mit $\sqrt{-1}$ nicht ignoriert werden konnten. 
Er führte systematische Rechenregeln\index{Rechenregeln} ein und zeigte,
dass man mit diesen Ausdrücken konsistent rechnen kann — sofern man sie
wie eigenständige Einheiten behandelt.

\begin{HistoryBox}[Bombellis entscheidender Schritt]
	Rafael (Raffaele) Bombelli erkannte um 1572, dass die merkwürdigen Ausdrücke 
	mit $\sqrt{-1}$ nicht ignoriert werden können. 
	Er führte systematische Rechenregeln ein und interpretierte $i^2=-1$
	als konsistente Erweiterung des bekannten Zahlensystems\index{Zahlensystem}. 
	Damit legte er den Grundstein für die Theorie der komplexen Zahlen.
\end{HistoryBox}

Bombelli selbst hatte noch keine geometrische Interpretation\index{Geometrie}. 
Aber er verstand: Wenn die Algebra\index{Algebra} solche Zahlen erfordert, müssen sie
ernst genommen werden.

\subsubsection*{Euler, Gauss und die mathematische Legitimation}
\phantomsection
Erst im 18.~und frühen 19.~Jahrhundert\index{19. Jahrhundert} erhielten die komplexen Zahlen
ihre volle mathematische Anerkennung.

Leonhard Euler\index{Euler, Leonhard} entdeckte den tiefen Zusammenhang zwischen komplexen
Zahlen und trigonometrischen Funktionen\index{Trigonometrische Funktionen}. Seine Formel
\[
e^{i\varphi} = \cos\varphi + i\sin\varphi
\]
zeigte, dass komplexe Zahlen keineswegs künstlich sind, sondern
natürlich in fundamentalen Beziehungen der Mathematik auftauchen.

Carl Friedrich Gauss\index{Gauss, Carl Friedrich} ging noch einen Schritt weiter. 
Er interpretierte komplexe Zahlen als Punkte in einer Ebene und gab ihnen damit eine
geometrische Bedeutung. Diese Gaußsche Zahlenebene\index{Gaußsche Zahlenebene} machte sichtbar, dass
komplexe Zahlen genauso „real“ sind wie reelle Zahlen — nur mit einer
zweiten Dimension\index{Zweite Dimension}.

Auch Jean-Robert Argand\index{Argand, Jean-Robert} trug wesentlich zur Verbreitung dieser
geometrischen Sichtweise bei.

\begin{MathBox}[Die grundlegende Beziehung]
	Die neue Zahl $i$ wird durch die Beziehung
	\[
	i^2 = -1
	\]
	definiert\index{$i^2=-1$}. Jede komplexe Zahl lässt sich als Kombination
	\[
	a + bi, \qquad a,b \in \mathbb{R},
	\]
	darstellen. Addition und Multiplikation bilden nach dieser Definition 
	einen vollständigen Körper\index{Körper (algebraisch)}. 
	Die geometrische Interpretation verbindet diese Darstellung mit der Ebene.
\end{MathBox}

\subsubsection*{Vom Rechenphänomen zur mathematischen Notwendigkeit}
\phantomsection
Mit der geometrischen Interpretation wurde klar: Komplexe Zahlen sind
keine willkürliche Erweiterung, sondern eine mathematische Notwendigkeit\index{Mathematische Notwendigkeit}.
Der \emph{Fundamentalsatz der Algebra}\index{Fundamentalsatz der Algebra} garantiert, 
dass jedes Polynom\index{Polynom} mindestens eine Lösung besitzt — und dieser Satz ist ohne komplexe Zahlen
nicht gültig.

Damit wurde der Zahlenbereich endgültig geschlossen: von den natürlichen
Zahlen\index{Natürliche Zahlen} über die ganzen\index{Ganze Zahlen} und reellen Zahlen\index{Reelle Zahlen} 
bis hin zu den komplexen Zahlen. 
Jede Erweiterung wurde von Problemen erzwungen, die ohne sie nicht vollständig lösbar waren.

Eine ausführliche mathematische Herleitung der komplexen Zahlen findet
sich im Anhang~\ref{anhangA_komplexe}. Dort werden die formale
Definition, die Rechenregeln und die geometrische Struktur vollständig
entwickelt.

\subsubsection*{Übergang}
\phantomsection
Die Geschichte der komplexen Zahlen zeigt, dass mathematische Strukturen
nicht erfunden werden, sondern aus der inneren Logik der Probleme
hervorgehen. Dieser Gedanke führt direkt zur nächsten Frage:
Warum funktioniert Mathematik überhaupt so zuverlässig?
Dies wird im nächsten Abschnitt behandelt.


%3.4
\subsection{Mathematik als System von Regeln}
\label{sec:3.4_regelsystem}
\phantomsection

\subsubsection*{Einleitung}
\phantomsection
Die bisherigen Entwicklungen – von negativen Zahlen\index{Negative Zahlen} 
bis zu den komplexen Zahlen\index{Komplexe Zahlen} – 
zeigen ein zentrales Muster: Die \emph{Mathematik}\index{Mathematik} 
wächst nicht durch willkürliche Erfindungen, sondern durch die innere Logik ihrer Regeln\index{Regeln}. 
Sobald man feste Rechenvorschriften akzeptiert, ergeben sich neue Strukturen\index{Struktur} zwangsläufig.

Mathematik ist deshalb kein Sammelsurium einzelner Tricks, sondern ein kohärentes 
System von klaren Regeln.

\subsubsection*{Regeln als Grundlage mathematischen Denkens}
\phantomsection
Seit der Antike\index{Antike} beruhen mathematische Methoden auf festen Operationen: 
Addition\index{Addition}, Subtraktion\index{Subtraktion}, 
Multiplikation\index{Multiplikation} und Division\index{Division}. 
Jede Erweiterung des Zahlbegriffs\index{Zahlbegriff} konnte nur bestehen, 
wenn diese Regeln konsistent blieben.

Das gilt für:
\begin{itemize}
	\item die Einführung der Null\index{Null},
	\item die Akzeptanz negativer Zahlen,
	\item die Irrationalität von $\sqrt{2}$\index{Irrationale Zahlen},
	\item die komplexen Zahlen $a + bi$,
	\item und spätere Strukturen wie Vektoren\index{Vektor} oder Funktionen\index{Funktion}.
\end{itemize}

Der entscheidende Punkt ist:
Sobald die Regeln definiert sind, dürfen sie angewendet werden – unabhängig davon, 
ob das Ergebnis intuitiv erscheint.

\subsubsection*{Die Idee der Axiome}
\phantomsection
Ein System von Regeln ist nur dann mächtig, wenn es auf eindeutigen Grundannahmen beruht. 
Solche Grundannahmen nennt man \emph{Axiome}\index{Axiom}. 
Sie legen fest, wie Objekte miteinander verknüpft werden dürfen, ohne Widersprüche zu erzeugen.

Bei den komplexen Zahlen etwa genügen wenige Vorgaben:
\begin{itemize}
	\item $i^2 = -1$\index{$i^2=-1$},
	\item reelle und imaginäre Teile\index{Reelle Zahlen} addieren sich getrennt,
	\item die bekannten Rechenregeln bleiben gültig.
\end{itemize}

Aus diesen wenigen Vorgaben entsteht ein vollständiger Zahlenbereich\index{Zahlenbereich}.  
Axiome begrenzen nicht – sie ermöglichen Struktur.

\subsubsection*{Regelsysteme erzeugen mathematische Welten}
\phantomsection
Sobald Regeln feststehen, kann man untersuchen, welche Objekte in diesem System existieren. 
Beispiele:

\begin{itemize}
	\item Aus den Regeln der Arithmetik\index{Arithmetik} entstehen die 
	natürlichen\index{Natürliche Zahlen}, ganzen\index{Ganze Zahlen} und 
	rationalen Zahlen\index{Rationale Zahlen}.
	\item Aus geometrischen Axiomen entsteht die euklidische Geometrie\index{Euklidische Geometrie}.
	\item Aus den Körperaxiomen\index{Körperaxiome} entwickelt sich die Analysis\index{Analysis}.
	\item Aus den Regeln der Algebra\index{Algebra} entstehen Polynomgleichungen\index{Polynomgleichung} 
	und deren Lösungen.
\end{itemize}

Die Mathematik entwickelt sich also von innen heraus – nicht durch äußere Entscheidungen.

\begin{DidacticBox}[Warum Regeln die Mathematik tragen]
	Mathematik wirkt manchmal wie eine Sammlung einzelner Themen – Zahlen, Gleichungen, 
	Funktionen oder Geometrie. Tatsächlich entsteht alles aus einem kleinen Satz klarer Regeln.
	
	Sind diese Regeln einmal festgelegt, folgen viele mathematische Strukturen unausweichlich.
	Deshalb „erfindet“ man neue Zahlen nicht, sondern entdeckt, dass sie notwendig sind, 
	um die Regeln widerspruchsfrei anzuwenden.
\end{DidacticBox}

\subsubsection*{Mathematik ist nicht erfunden, sondern entdeckt}
\phantomsection
Die Geschichte der Zahlen\index{Zahlen} zeigt ein klares Muster:
Keine Zahl wurde eingeführt, weil jemand sie erfinden wollte. Jede neue Zahl entstand, 
weil die bestehenden Regeln sie erzwungen haben. 
Negative Zahlen, irrationale Zahlen und komplexe Zahlen sind nicht Produkte der Fantasie, 
sondern Konsequenzen der mathematischen Struktur.

\subsubsection*{Übergang}
\phantomsection
Wenn mathematische Regeln eine konsistente Welt\index{Welt} erzeugen, 
stellt sich die Frage, warum diese Welt die Struktur der Natur widerspiegelt. 
Diese Verbindung wird im nächsten Kapitel untersucht.



