\chapter{Mathematik als Sprache der Natur}
\label{chap:IV_sprache}
\label{chap:IV_algorithmen}
\setcounter{section}{4}
\setcounter{subsection}{0}
\setcounter{subsubsection}{1}
\setcounter{secnumdepth}{3}
\setlength{\parindent}{0pt}


\subsection{Galilei und die neue Physik}
\textbf{Galilei}\index{Galilei, Galileo} erkannte, dass die Naturgesetze mathematisch formulierbar sind. 
Seine Experimente mit fallenden Körpern machten deutlich: Bewegung folgt klaren Regeln, die sich in Gleichungen ausdrücken lassen. 

\subsection{Funktionen und Bewegungsgesetze}
Mit dem Begriff der \textbf{Funktion}\index{Funktion} konnten Zusammenhänge präzise beschrieben werden. 
Ob Geschwindigkeit, Kraft oder Energie – in jedem Fall wurde eine mathematische Form die Sprache der Natur. 

\subsection{Differentialgleichungen und Wachstum}
Die Einführung der \textbf{Differentialgleichungen}\index{Differentialgleichung} eröffnete neue Möglichkeiten. 
Von der Mechanik über die Himmelskörper bis hin zu Wachstumsprozessen ließen sich Entwicklungen nun systematisch erfassen. 

\subsection{Von der Mechanik zur modernen Physik}
Die Sprache der Mathematik blieb nicht auf die klassische Mechanik beschränkt. 
Auch Elektrodynamik, Relativitätstheorie und Quantenphysik beruhen auf denselben Prinzipien – 
die Natur spricht überall dieselbe Sprache. 
