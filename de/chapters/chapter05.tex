\chapter{Wahrscheinlichkeiten und Zufall}
\label{chap:V_wahrscheinlichkeit}
\label{chap:V_realisationen}
\setcounter{section}{5}
\setcounter{subsection}{0}
\setcounter{subsubsection}{1}
\setcounter{secnumdepth}{3}
\setlength{\parindent}{0pt}


\subsection{Würfel und die Entstehung der Wahrscheinlichkeitstheorie}
Das Spiel mit Würfeln war mehr als Unterhaltung – es führte zur Frage, 
wie man den Zufall mathematisch beschreiben kann. 
Hier entstand die Wahrscheinlichkeitstheorie als eigenständiges Gebiet. 

\subsection{Statistik und Naturgesetze}
Mit der Statistik lernte man, viele Einzelereignisse zu ordnen und zu mitteln. 
So konnten Naturgesetze sichtbar werden, die sich nur in großen Zahlen offenbaren. 

\subsection{Quantenmechanik und Zufall}
In der Quantenmechanik ist der Zufall nicht nur ein Mangel an Wissen, 
sondern ein grundlegendes Prinzip der Natur. 
Die Mathematik der Wahrscheinlichkeiten ist hier Teil der Wirklichkeit selbst. 

\subsection{Ordnung im Chaos}
Selbst in scheinbar chaotischen Prozessen zeigen sich Strukturen. 
Die moderne Mathematik des Chaos und der nichtlinearen Systeme 
macht deutlich, dass Zufall und Ordnung eng verbunden sind. 
