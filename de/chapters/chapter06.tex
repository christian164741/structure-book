\chapter{Abstraktion und Struktur}
\label{chap:VI_abstraktion}

\setcounter{section}{6}
\setcounter{subsection}{0}
\setcounter{subsubsection}{1}
\setcounter{secnumdepth}{3}


\subsection{Vektoren und Räume}
Mit der Einführung von \textbf{Vektoren}\index{Vektoren} und abstrakten \textbf{Räumen}\index{Räume} 
konnte man geometrische und physikalische Probleme auf eine neue Ebene heben. 
Damit begann die Mathematik, Strukturen unabhängig von Anschauung zu betrachten. 

\subsection{Symmetrien und Invarianz}
\textbf{Symmetrien}\index{Symmetrien} spielen in Mathematik und Physik eine fundamentale Rolle. 
Sie zeigen, welche Eigenschaften eines Systems sich nicht ändern – auch wenn sich die Darstellung wandelt. 

\subsection{Mathematik als universelles Strukturprinzip}
Immer deutlicher wurde, dass Mathematik nicht nur Werkzeuge liefert, 
sondern selbst die Form der Naturgesetze bestimmt. 
Abstrakte Prinzipien wie Invarianz oder Transformationen prägen unser Verständnis der Welt. 

\subsection{Abstraktion als Weg zur Wirklichkeit}
Was wie reine Gedankenarbeit wirkt, erweist sich oft als Beschreibung realer Phänomene. 
Abstraktion ist damit kein Gegensatz zur Realität, 
sondern ein Weg, deren Struktur sichtbar zu machen. 
