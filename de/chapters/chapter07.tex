\chapter{Wie komplexe Zahlen Struktur erzeugen}
\label{chap:VII_komplexe Struktur}

\setcounter{section}{7}
\setcounter{subsection}{0}
\setcounter{subsubsection}{1}
\setcounter{secnumdepth}{3}
% Boxen-Stile definieren




% ================================
% Teil III – Mathematik und Philosophie
% ================================



\subsection{Iteration als mathematisches Prinzip}
Für Platon war Mathematik ein Teil der ewigen Ideenwelt. 
Zahlen und geometrische Formen existierten unabhängig vom Menschen. 

\subsection{Stabilität, Divergenz und Grenzfälle}
Aristoteles betonte den praktischen Nutzen der Mathematik – 
als Werkzeug für Logik, Physik und Naturbeschreibung. 

\subsection{Die Mandelbrot-Menge als Landkarte}
Kant sah in der Mathematik eine apriorische Form unseres Denkens: 
Raum und Zeit sind die Grundlage, auf der wir Naturgesetze erkennen können. 

\subsection{Julia-Mengen: lokale Dynamik und Struktur}
Heute wird die Frage neu gestellt: 
Ist Mathematik eine menschliche Erfindung, ein nützliches Sprachspiel – 
oder eine Entdeckung von Strukturen, die unabhängig von uns existieren? 
Philosophie und Naturwissenschaft bleiben hier eng verbunden. 
\subsection{Selbstähnlichkeit und Skaleninvarianz}
Heute wird die Frage neu gestellt: 

\subsection{Warum diese Formen keine Naturmodelle sind}
Heute wird die Frage neu gestellt: 

\subsection{	Fazit: Struktur ohne Materie}
Heute wird die Frage neu gestellt: 