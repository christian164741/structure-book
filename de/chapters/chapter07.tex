git\chapter{Mathematik und Philosophie}
\label{chap:VII_philosophie}
\label{chap:VII_anwendungen}
\setcounter{section}{7}
\setcounter{subsection}{0}
\setcounter{subsubsection}{1}
\setcounter{secnumdepth}{3}
% Boxen-Stile definieren




% ================================
% Teil III – Mathematik und Philosophie
% ================================



\subsection{Platon und die Ideenwelt}
Für Platon war Mathematik ein Teil der ewigen Ideenwelt. 
Zahlen und geometrische Formen existierten unabhängig vom Menschen. 

\subsection{Aristoteles und die praktische Mathematik}
Aristoteles betonte den praktischen Nutzen der Mathematik – 
als Werkzeug für Logik, Physik und Naturbeschreibung. 

\subsection{Kant und die Bedingungen der Erkenntnis}
Kant sah in der Mathematik eine apriorische Form unseres Denkens: 
Raum und Zeit sind die Grundlage, auf der wir Naturgesetze erkennen können. 

\subsection{Moderne Positionen}
Heute wird die Frage neu gestellt: 
Ist Mathematik eine menschliche Erfindung, ein nützliches Sprachspiel – 
oder eine Entdeckung von Strukturen, die unabhängig von uns existieren? 
Philosophie und Naturwissenschaft bleiben hier eng verbunden. 
