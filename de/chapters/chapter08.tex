
\chapter{Mathematik im 21. Jahrhundert}
\label{chap:VIII_21jh}
\label{chap:VII_philosophie}
\label{chap:VIII_ausblick}
\setcounter{section}{8}
\setcounter{subsection}{0}
\setcounter{subsubsection}{1}
\setcounter{secnumdepth}{3}
% ================================
% Teil IV – Mathematik im 21. Jahrhundert
% ================================



\subsection{Gödel und die Grenzen formaler Systeme}
Mit seinen Unvollständigkeitssätzen zeigte \textbf{Gödel, Kurt}\index{Gödel, Kurt}, 
dass es in jedem formalen System wahre Aussagen gibt, die nicht beweisbar sind. 
Dies veränderte das Verständnis von Mathematik grundlegend. 

\subsection{Computer und künstliche Intelligenz}
Computer haben die Mathematik praktisch und theoretisch revolutioniert. 
Von automatisierten Beweisen bis hin zu \textbf{künstlicher Intelligenz}\index{Künstliche Intelligenz} 
stellt sich die Frage: Lösen Maschinen nur Aufgaben – oder entdecken sie selbst Strukturen? 

\subsection{Mathematik und moderne Physik}
Die heutige Physik – von der Relativitätstheorie bis zur Quantenfeldtheorie – 
zeigt, dass Mathematik nicht nur Werkzeug ist, sondern den Rahmen der Naturgesetze vorgibt. 

\subsection{Ausblick: Die Struktur der Welt}
Die Reise führt zu einer offenen Frage: 
Ist Mathematik die Sprache, die wir erfinden – oder die Struktur, die wir entdecken? 
Vielleicht ist sie beides zugleich: ein Spiegel des Geistes und der Welt. 
