\chapter{Mathematik im 21. Jahrhundert}
\label{chap:VIII_21jh}
\label{chap:VII_philosophie}
\label{chap:VIII_ausblick}




\subsection{Gödel und die Grenzen formaler Systeme}
Mit seinen Unvollständigkeitssätzen zeigte \textbf{Gödel, Kurt}\index{Gödel, Kurt}, 
dass es in jedem formalen System wahre Aussagen gibt, die nicht beweisbar sind. 
Dies veränderte das Verständnis von Mathematik grundlegend. 

\subsection{Computer und künstliche Intelligenz}
Computer haben die Mathematik praktisch und theoretisch revolutioniert. 
Von automatisierten Beweisen bis hin zu \textbf{künstlicher Intelligenz}\index{Künstliche Intelligenz} 
stellt sich die Frage: Lösen Maschinen nur Aufgaben – oder entdecken sie selbst Strukturen? 

\subsection{Mathematik und moderne Physik}
Die heutige Physik – von der Relativitätstheorie bis zur Quantenfeldtheorie – 
zeigt, dass Mathematik nicht nur Werkzeug ist, sondern den Rahmen der Naturgesetze vorgibt. 

\subsection{Ausblick: Die Struktur der Welt}
Die Reise führt zu einer offenen Frage: 
Ist Mathematik die Sprache, die wir erfinden – oder die Struktur, die wir entdecken? 
Vielleicht ist sie beides zugleich: ein Spiegel des Geistes und der Welt. 
\subsection{Methodentransfer: String-Mathematik in realen Netzwerken}
\index{Stringtheorie}
\index{Minimalfl\"achen}
\index{Optimierung}
\index{Netzwerk}
\index{Biologie}

Im 21.~Jahrhundert ist die spannendste Rolle der Mathematik oft nicht mehr die Erfindung eines \glqq neuen\grqq\ Rechenverfahrens,
sondern der Transfer eines bereits entwickelten Werkzeugkastens in ein ganz anderes Gebiet.
Ein besonders anschauliches Beispiel ist der Einsatz von Methoden aus der Stringtheorie, um die Geometrie realer Verzweigungsnetzwerke
zu beschreiben: Nervenzellen, Blutgef\"a{\ss}e, Korallen -- oder auch B\"aume.

\begin{DidacticBox}[Nicht Strings im Gehirn -- sondern String-Mathe als Werkzeug]
	\index{Minimalfl\"ache}
	\index{Verzweigung}
	\index{R\"ohrennetzwerk}
	Die Aussage lautet \emph{nicht}: \glqq Das Gehirn besteht aus Strings.\grqq\ 
	Sondern: Viele biologische Strukturen sind \emph{R\"ohrennetzwerke} mit endlicher Dicke.
	Dann ist nicht nur die Gesamtl\"ange entscheidend, sondern vor allem die ben\"otigte \emph{Oberfl\"ache} (Material, Erhalt, Stabilit\"at)
	sowie der Aufwand an Verzweigungen (lokale \glqq Nahtstellen\grqq).
	Mathematisch f\"uhrt das zu einem Minimalfl\"achen- bzw.\ Fl\"achen-Optimierungsproblem --
	und genau f\"ur solche Probleme stellt die Stringtheorie seit Jahrzehnten leistungsf\"ahige Methoden bereit.
\end{DidacticBox}
\label{box:string-mathe-netzwerke}

Ein intuitives Bild liefert die Botanik:
Ein Baum maximiert die \emph{Lichtsammlung} \"uber seine Blattfl\"ache, muss aber gleichzeitig Wasser und N\"ahrstoffe
\"uber ein verzweigtes Leitungsnetz nach oben transportieren.
Die entstehende Architektur ist ein Kompromiss aus Sammeln, Transport, Materialaufwand und Stabilit\"at --
und genau solche Kompromisse lassen sich als Optimierungsprobleme modellieren.
\index{Baum}
\index{Evolution}