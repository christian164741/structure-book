\cleardoublepage
\thispagestyle{empty}

\begin{flushleft}
	\begin{tabular}{@{}l l@{}}
		\textbf{Titel:} &
		\parbox[t]{0.65\textwidth}{\textit{Die Struktur der Welt}}\\[0.4em]
		
		\textbf{Untertitel:} &
		\parbox[t]{0.65\textwidth}{Mathematik zwischen Abstraktion und Realität}\\[1.2em]
		
		\textbf{Autor:} & Dipl.-Ing.\,(FH) Christian Weilharter\\[0.5em]
		\textbf{\copyright\ 2026} & Christian Weilharter, Traunstein\\[0.5em]
		\textbf{ISBN:} & 978-3-912302-11-0\\
		\textbf{Satz:} & \LaTeX\\
		\textbf{Druck:} & [Print-on-Demand-Dienst]\\
		\textbf{Kontakt:} & christian@weilharter.de\\
		\textbf{Web:} & www.mathandphysics.de\\
	\end{tabular}
\end{flushleft}

\vspace{2em}
\noindent
Alle Rechte vorbehalten. Kein Teil dieses Buches darf ohne schriftliche Genehmigung des Autors 
in irgendeiner Form reproduziert, gespeichert oder übertragen werden, 
weder elektronisch, mechanisch, durch Fotokopien, Aufnahmen noch auf andere Weise.
\setlength{\parindent}{0pt}

\chapter*{Vorwort}
\addcontentsline{toc}{chapter}{Vorwort}

Mathematik erscheint oft als eine Sammlung von Formeln und Symbolen – 
etwas, das man in der Schule lernen muss, ohne dass klar wird, warum. 
Doch sie ist weit mehr: ein inneres Gerüst, eine unsichtbare Struktur, 
die unsere Welt trägt und formt.

Die Naturwissenschaften haben immer wieder gezeigt, dass ihre Gesetze 
nicht frei erfunden sind, sondern sich aus mathematischen Strukturen ergeben, 
die wir entdecken. Ob es die Kreiszahl~$\pi$ ist, die seit der Antike 
in Geometrie und Physik auftaucht, die komplexen Zahlen, die zur Grundlage 
der Quantenmechanik wurden, oder die Wahrscheinlichkeitsrechnung, 
die unser Verständnis von Zufall verändert hat – überall zeigt sich: 
Mathematik ist Teil der Realität.

Dieses Buch möchte diese Doppelrolle sichtbar machen: 
als \textbf{Abstraktion} im Denken und als \textbf{Realität} in Naturgesetzen. 
Es erzählt, wie Ideen entstehen, wie sie sich entwickeln und wie sie 
zeigen, dass Mathematik und Wirklichkeit nicht zu trennen sind.

\vspace{2em}

\begin{flushright}
	\textit{Dipl.-Ing.(FH) Christian Weilharter} \\
	\vspace{0.5em}
	[Traunstein, 2025]
\end{flushright}
