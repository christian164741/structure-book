\cleardoublepage
\thispagestyle{empty}

\begin{flushleft}
	\begin{tabular}{@{}l l@{}}
		\textbf{Titel:} &
		\parbox[t]{0.65\textwidth}{\textit{Die Struktur der Welt}}\\[0.4em]
		
		\textbf{Untertitel:} &
		\parbox[t]{0.65\textwidth}{Mathematik zwischen Abstraktion und Realität}\\[1.2em]
		
		\textbf{Autor:} & Dipl.-Ing.\,(FH) Christian Weilharter\\[0.5em]
		\textbf{\copyright\ 2026} & Christian Weilharter, Traunstein\\[0.5em]
		\textbf{ISBN:} & 978-3-912302-11-0\\
		\textbf{Satz:} & \LaTeX\\
		\textbf{Druck:} & [Print-on-Demand-Dienst]\\
		\textbf{Kontakt:} & christian@weilharter.de\\
		\textbf{Web:} & www.mathandphysics.de\\
	\end{tabular}
\end{flushleft}

\vspace{2em}
\noindent
Alle Rechte vorbehalten. Kein Teil dieses Buches darf ohne schriftliche Genehmigung des Autors 
in irgendeiner Form reproduziert, gespeichert oder übertragen werden, 
weder elektronisch, mechanisch, durch Fotokopien, Aufnahmen noch auf andere Weise.
\setlength{\parindent}{0pt}

\chapter*{Vorwort}
\addcontentsline{toc}{chapter}{Vorwort}

Mathematik erscheint oft als eine Sammlung von Formeln und Symbolen --
etwas, das man lernt, ohne dass sofort klar wird, wozu es gut ist.
Doch Mathematik ist weit mehr: ein inneres Gerüst, eine unsichtbare Struktur,
mit der wir die Welt ordnen, verstehen und vorhersagen.

Die Naturwissenschaften zeigen immer wieder, dass ihre Gesetze nicht willkürlich sind.
Sie folgen aus mathematischen Beziehungen, die wir nicht erfinden, sondern freilegen.
Ob es die Kreiszahl~$\pi$ ist, die seit der Antike in Geometrie und Physik auftaucht,
die komplexen Zahlen, die zur Sprache der Quantenmechanik wurden,
oder die Wahrscheinlichkeitsrechnung, die unser Verständnis von Zufall präzisiert --
überall begegnen wir derselben Erfahrung: Mathematik trifft reale Strukturen.

Dieses Buch möchte diese Doppelrolle sichtbar machen:
Mathematik als \textbf{Abstraktion} des Denkens und als \textbf{Struktur} der Natur.
Es erzählt, wie Ideen entstehen, wie sie sich entwickeln und wie sie zeigen,
dass Mathematik und Wirklichkeit tiefer verbunden sind, als es auf den ersten Blick scheint.

\vspace{2em}

\begin{flushright}
	\textit{Dipl.-Ing.\,(FH) Christian Weilharter}\\
	\vspace{0.5em}
	Traunstein, 2026
\end{flushright}
\newpage
\noindent
\begin{NoteBox}[Wie dieses Buch gelesen werden kann]
	\small
	Dieses Buch ist als roter Faden gebaut: Begriffe werden eingeführt, an Beispielen getestet und am Ende jedes Kapitels verdichtet.
	Boxen markieren Schlüsselpunkte (Definitionen, Denkfehler, Kernaussagen).
	Der Fließtext trägt die Argumentation; die Boxen sind Orientierung, nicht Ersatz.
\end{NoteBox}